\documentclass{beamer}					% Document class

\usepackage[spanish]{babel}				% Set language
\usepackage[utf8x]{inputenc}			% Set encoding
\usepackage{graphicx}					% For including figures
\usepackage{booktabs}					% For table rules
\usepackage{hyperref}					% For cross-referencing


\title{Introducción a Github}
\author{Tom\'as Pacheco}	
\institute{Econom\'ia Aplicada - Universidad de San Andr\'es}			
\date{5 de Agosto 2021}							

\begin{document}

\begin{frame}
  \titlepage
\end{frame}

\begin{frame}{¿Qué es Github?}

\begin{itemize}
    \item Es una plataforma que guarda c\'odigos para el control de versiones y colaboraci\'on. Permite trabajar en coolaboraci\'on con otras personas desde cualquier lugar. 
\end{itemize}
\end{frame}

\begin{frame}{Para comenzar}

El objetivo de esta clase es que Github les sirva para hacer los Problem Sets de la materia. Vamos a trabajar con Github Desktop.
\begin{itemize}
    \item Crear una cuenta de \href{https://bit.ly/3kuoEd4}{Github} 
    \item Descargar \href{https://desktop.github.com/}{Github Desktop}
\end{itemize}  
\end{frame}

\begin{frame}{Logo}

\begin{center}
imagen logo
\end{center}

\end{frame}



\begin{frame}{}
\begin{huge}
\begin{center}
    tpacheco@udesa.edu.ar
\end{center}
\end{huge}
\end{frame}

\end{document}
